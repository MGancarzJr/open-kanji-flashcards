% kanji_template.tex
% This is the template file for all single-kanji flashcards and needs to be included via % kanji_template.tex
% This is the template file for all single-kanji flashcards and needs to be included via % kanji_template.tex
% This is the template file for all single-kanji flashcards and needs to be included via % kanji_template.tex
% This is the template file for all single-kanji flashcards and needs to be included via \input{kanji_template} at the preamble.
% Editing this file will alter the look and function of all generated cards.

\documentclass[2col4row,grid]{flashcards}

\cardfrontstyle[\Large]{headings}
\cardbackstyle{empty}

\usepackage{xeCJK}
\setCJKsansfont{MS Gothic} % for \sffamily

\usepackage{multirow}

\newcounter{cardnum}

%%%%%%%%%%%%%%%%%%%%%%%%%%%%%%%%%%%
% Macro definition for card layout. This takes 13 parameters in the format:
%
% \shortcard{kanji}
%			{definition}
%			{reading}
%			{example 1}{definition and reading 1}
%			...
%			{example 5}{definition and reading 5}
%%%%%%%%%%%%%%%%%%%%%%%%%%%%%%%%%%%

% Provide a way to declare and renew a command in one command. Black magic provided by Jämes.
\newcommand{\neworrenewcommand}[1]{\providecommand{#1}{}\renewcommand{#1}}

\newcommand{\shortcard}[9]{
	\neworrenewcommand{\sshortcard}[4]{
		\begin{flashcard}[\arabic{cardnum}]{\begin{tabular}{ c l }
		\multirow{5}{6em}{\fontsize{90}{108}#1} & #4 \\ 
		& #6 \\ 
		& #8 \\
		& ##1 \\
		& ##3
		\end{tabular}}

		{\large #2 \\
		#3}

		\smallskip

		\begin{description}
			\item #4 • #5
			\item #6 • #7
			\item #8 • #9
			\item ##1 • ##2
			\item ##3 • ##4
		\end{description}

		\end{flashcard}
		\stepcounter{cardnum}
	}
	\sshortcard
}
% End macro at the preamble.
% Editing this file will alter the look and function of all generated cards.

\documentclass[2col4row,grid]{flashcards}

\cardfrontstyle[\Large]{headings}
\cardbackstyle{empty}

\usepackage{xeCJK}
\setCJKsansfont{MS Gothic} % for \sffamily

\usepackage{multirow}

\newcounter{cardnum}

%%%%%%%%%%%%%%%%%%%%%%%%%%%%%%%%%%%
% Macro definition for card layout. This takes 13 parameters in the format:
%
% \shortcard{kanji}
%			{definition}
%			{reading}
%			{example 1}{definition and reading 1}
%			...
%			{example 5}{definition and reading 5}
%%%%%%%%%%%%%%%%%%%%%%%%%%%%%%%%%%%

% Provide a way to declare and renew a command in one command. Black magic provided by Jämes.
\newcommand{\neworrenewcommand}[1]{\providecommand{#1}{}\renewcommand{#1}}

\newcommand{\shortcard}[9]{
	\neworrenewcommand{\sshortcard}[4]{
		\begin{flashcard}[\arabic{cardnum}]{\begin{tabular}{ c l }
		\multirow{5}{6em}{\fontsize{90}{108}#1} & #4 \\ 
		& #6 \\ 
		& #8 \\
		& ##1 \\
		& ##3
		\end{tabular}}

		{\large #2 \\
		#3}

		\smallskip

		\begin{description}
			\item #4 • #5
			\item #6 • #7
			\item #8 • #9
			\item ##1 • ##2
			\item ##3 • ##4
		\end{description}

		\end{flashcard}
		\stepcounter{cardnum}
	}
	\sshortcard
}
% End macro at the preamble.
% Editing this file will alter the look and function of all generated cards.

\documentclass[2col4row,grid]{flashcards}

\cardfrontstyle[\Large]{headings}
\cardbackstyle{empty}

\usepackage{xeCJK}
\setCJKsansfont{MS Gothic} % for \sffamily

\usepackage{multirow}

\newcounter{cardnum}

%%%%%%%%%%%%%%%%%%%%%%%%%%%%%%%%%%%
% Macro definition for card layout. This takes 13 parameters in the format:
%
% \shortcard{kanji}
%			{definition}
%			{reading}
%			{example 1}{definition and reading 1}
%			...
%			{example 5}{definition and reading 5}
%%%%%%%%%%%%%%%%%%%%%%%%%%%%%%%%%%%

% Provide a way to declare and renew a command in one command. Black magic provided by Jämes.
\newcommand{\neworrenewcommand}[1]{\providecommand{#1}{}\renewcommand{#1}}

\newcommand{\shortcard}[9]{
	\neworrenewcommand{\sshortcard}[4]{
		\begin{flashcard}[\arabic{cardnum}]{\begin{tabular}{ c l }
		\multirow{5}{6em}{\fontsize{90}{108}#1} & #4 \\ 
		& #6 \\ 
		& #8 \\
		& ##1 \\
		& ##3
		\end{tabular}}

		{\large #2 \\
		#3}

		\smallskip

		\begin{description}
			\item #4 • #5
			\item #6 • #7
			\item #8 • #9
			\item ##1 • ##2
			\item ##3 • ##4
		\end{description}

		\end{flashcard}
		\stepcounter{cardnum}
	}
	\sshortcard
}
% End macro at the preamble.
% Editing this file will alter the look and function of all generated cards.

\documentclass[2col4row,grid]{flashcards}

\cardfrontstyle[\Large]{headings}
\cardbackstyle{empty}

\usepackage{xeCJK}
\setCJKsansfont{MS Gothic} % for \sffamily

\usepackage{multirow}

\newcounter{cardnum}

%%%%%%%%%%%%%%%%%%%%%%%%%%%%%%%%%%%
% Macro definition for card layout. This takes 13 parameters in the format:
%
% \shortcard{kanji}
%			{definition}
%			{reading}
%			{example 1}{definition and reading 1}
%			...
%			{example 5}{definition and reading 5}
%%%%%%%%%%%%%%%%%%%%%%%%%%%%%%%%%%%

% Provide a way to declare and renew a command in one command. Black magic provided by Jämes.
\newcommand{\neworrenewcommand}[1]{\providecommand{#1}{}\renewcommand{#1}}

\newcommand{\shortcard}[9]{
	\neworrenewcommand{\sshortcard}[4]{
		\begin{flashcard}[\arabic{cardnum}]{\begin{tabular}{ c l }
		\multirow{5}{6em}{\fontsize{90}{108}#1} & #4 \\ 
		& #6 \\ 
		& #8 \\
		& ##1 \\
		& ##3
		\end{tabular}}

		{\large #2 \\
		#3}

		\smallskip

		\begin{description}
			\item #4 • #5
			\item #6 • #7
			\item #8 • #9
			\item ##1 • ##2
			\item ##3 • ##4
		\end{description}

		\end{flashcard}
		\stepcounter{cardnum}
	}
	\sshortcard
}
% End macro